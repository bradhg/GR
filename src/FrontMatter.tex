\documentclass[GR.tex]{subfiles}

\begin{document}
    %-----------------------------------------------------------------------------------------------
    \begin{titlepage}
        \centering
        \vspace*{3cm}  % push down from top of page
        {\Huge\bfseries Independent Study of General Relativity\par}
        \vspace{1.5cm}
        {\huge Brad Garn\par}
        \vspace{0.5cm}
        {\large\today\par}
        \vfill
    \end{titlepage}

    \setcounter{tocdepth}{2}
    \tableofcontents

    %===============================================================================================
    \newpage

    %-----------------------------------------------------------------------------------------------
    \section*{Introduction}
    This is my personal record of studying general relativity.
    I am writing it to:
    \begin{itemize}[leftmargin=2em]
        \item organize the material in a way that makes sense to me,
        \item work through the derivations step by step,
        \item force myself to understand the material more deeply by writing it out, and
        \item create a reference I can return to as I continue learning.
    \end{itemize}
    This is not intended as a textbook or a guide for others.
    It’s simply my personal working notes.
    It is still a work in progress.
    Many sections are incomplete, and others will change as I learn more and expand my knowledge.

    %-----------------------------------------------------------------------------------------------
    \section*{Acknowledgments}

    In my years of schooling I had many math and science teachers.
    Even though I can no longer remember most of their names, I appreciate all that they taught me.
    With respect to calculus and physics, there were two teachers at Mountain View High School in Mesa, AZ who helped spark my interest in these subjects:
    \begin{itemize}[leftmargin=2em]
        \item Mr.\ Rice — AP Physics
        \item Mr.\ Slade — Trigonometry \& Math Analysis, and AP Calculus
    \end{itemize}

    The EigenChris YouTube channel has been an important learning resource.
    I have tried to study this topic in the past and was never able to get a grasp of tensors from just reading textbooks.
    Its three playlists \emph{Tensors for Beginners}, \emph{Tensor Calculus}, and \emph{Relativity by EigenChris} have been especially helpful.

    ChatGPT has provided on-demand explanations, clarifications, and guidance while I worked through the mathematics and physics of general relativity.

    I am likewise grateful for the many scientists and mathematicians whose work forms the foundation of general relativity, including Albert Einstein, Karl Schwarzschild, Bernhard Riemann, Elwin Christoffel, Tullio Levi-Civita, and Hermann Minkowski.

    \vfill
    \begin{center}
        \small
        PDF: \href{https://BradGarn.com/GR.pdf}{BradGarn.com/GR.pdf}\\
        Source: \href{https://github.com/bradhg/GR}{github.com/bradhg/GR}\\
        \begin{luacode*}
        -- print current git SHA
        local f = io.popen("git describe --always --dirty --abbrev=7 2>nul")

        if f then
            local line = f:read("*l")
            f:close()
            if line and line ~= "" then
                tex.print("Git SHA: "..line)
            end
        end
        \end{luacode*}

    \end{center}
\end{document}
